\documentclass[a4paper, 11pt]{article}
\usepackage{amsthm, amsmath, amsfonts, amssymb}
\usepackage{geometry}

% Page layout settings
\geometry{margin=1in}

\begin{document}

\subsection*{(a) Induktionsbeweis}

\begin{center}
    \begin{minipage}{0.4\textwidth}
    $g_1(n) = 5 + g_1(n - 1) + g_1(n - 2)$ \\
    $g_2(n) = 6 + 6 (n - 2) = 6n - 6$
    \end{minipage}
    \end{center}

\noindent INDUKTIONSANFANG: $n = 5$ \\
    Behauptung: $g_1(5) > g_2(5)$ \\
    Beweis: \\
    Es gilt: \hfill $g_1(5) = 5 + 18 + 10 = 33, \quad g_2(5) = 6 \cdot 5 - 6 = 24$ \hfill also: \quad $g_1(5) > g_2(5)$

\vspace{0.5cm}
    
\noindent INDUKTIONSSCHRITT: $n \rightarrow n + 1$ \\
Sei $n \in \mathbb{N}$ und $n \geq 5$ beliebig.

\noindent Induktionsvoraussetzung: 
    $$g_1(n) > g_2(n)$$

\noindent Behauptung: $g_1(n + 1) > g_2(n + 1)$ \quad also \quad $5 + g_1(n) + g_1(n - 1) > 6(n + 1) - 6$ \quad also \quad $g_1(n + 1) > 6n$

\noindent Beweis: $5 + g_1(n) + g_1(n - 1) > 6n$
\begin{align}
    &g_1(n + 1) = 5 + g_1(n) + g_1(n - 1) \\
    &g_1(n) > 6n - 6, \hspace{1cm} g_1(n - 1) > 6n - 12 \quad \text{(nach IV)} \\
    &g_1(n + 1) > 5 + 6n - 6 + 6n - 12 \\
    &g_1(n + 1) > 12n - 13 \\
    &g_1(n + 1) > 12n - 13 > 6n \iff 6n > 13
\end{align}

\noindent Insgesamt bewiesen: \\
$g_1(n) > g_2(n)$ f.a. $n \in \mathbb{N}$ mit $n \geq 5$

\hfill \qed
\vspace{0.5cm}
\hrule





\end{document}

