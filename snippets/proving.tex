\documentclass[a4paper, 11pt]{article}
\usepackage{amsthm, amsmath, amsfonts, amssymb}
\usepackage{geometry}

% Page layout settings
\geometry{margin=1in}

\begin{document}

\subsection*{(b)}
\subsubsection*{(i) \normalfont Wenn f und g injektiv sind, so ist auch h injektiv.}


$f(x_1) = f(x_2) \implies x_1 = x_2$ \quad f.a. $x_1, x_2 \in A$ \\
$g(y_1) = g(y_2) \implies y_1 = y_2$ \quad f.a. $y_1, y_2 \in B$
\begin{align}
    h := g(f(x)) \rightarrow g(f(x_1)) &= g(f(x_2)) \\
    f(x_1) &= f(x_2) \\
    x_1 &= x_2
\end{align}

$\rightarrow h$ muss ebenfalls injektiv sein. \hfill \qed



\subsubsection*{(ii) \normalfont Wenn $h$ surjektiv und $g$ injektiv, dann ist $g$ bijektiv.}

$g$ ist injektiv: f.a. $z \in C$ ex. höchstens ein $y \in B : g(y) = z$

\noindent h ist surjektiv: f.a. $z \in C$ ex. mindestens ein $x \in A : h(x) = z$ \\

\noindent Zu zeigen: $g$ ist surjektiv d.h.

\noindent f.a. $z \in C$ ex. mindestens ein $y \in B : g(y) = z$
\setcounter{equation}{0}
\begin{align}
    h(x) = g(f(x)) = z \\
    f(x) = y \text{ und } y \in B \\
    g(y) = z \iff h(x) = z
\end{align}

\noindent Da $h(x)$ surjektiv ist, muss auch $g(y)$ surjektiv sein. $g$ ist also bijektiv (Injektivität geht aus der Aufgabenstellung hervor).



\hfill \qed
\vspace{0.5cm}
\hrule

\end{document}
